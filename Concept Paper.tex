\documentclass{article}

\begin{document}

\title{A MACHINE LEARNING APPROACH TO PERSONALIZING EDUCATION: IMPROVING INDIVIDUAL LEARNING THROUGH TRACKING AND COURSE RECOMMENDATION}
\author{}
\maketitle

\section{Introduction}
This document contains information relating the skill of individual learning through tracking and course recommendation which students in the computer science field should undertake so as to acquire relevant skills.
\subsection{Abstract}
Accurately predicting students’ future performance based on their ongoing academic records is crucial for effectively carrying out necessary interventions to ensure students’
on-time and satisfactory graduation. Although there is a rich literature on predicting student performance when solving problems or studying for courses using data-driven approaches,
predicting student performance in completing degrees (e.g. college programs) is much less studied.

\subsection{Background}
There are steadily increasing numbers of students enrolling in college and universities but the degrees offered ignore differences in backgrounds, abilities, learning styles and career goals. Worse more, due to insufficient teachers, students have a lot of work to  do by themselves. Poor learning results as well as low engagement, dissatisfaction and high dropout rates. In this project, an interactive electronic system will be built that is personalized for each student, is able to continuously track progress and goals, capitalize on the knowledge accumulated, and recommend suitable courses and activities in order to build skills, enhance interest and promote  long-term goals. In effect, our personalized interactive system operates as “if” there is a dedicated mentor for each student. To build this system, the following modules will need to be developed: (1) student and course similarity discovery methods; (2) student performance prediction algorithms; (3) personalized course recommendation algorithms.

\section{Objectives}
To find out how machine learning through softwares that track students' performance can determine the courses they can do best and improve on their performance in education.

\section{Problem Statement}
If you ask a random group of students say in University if they like the Programs they are offering and performing well in those respective courses, they will most likely say that they are doing fair or hopefully they will get better. In fact, most of them are actually offering them because of external influence but not by own will. So students are forced to read hard sometimes for not the right programs and end up not doing well due to lack of a tracking approach in their studies.

\section{Methodology}
We have used a literature review in our study. Our research is both descriptive and explanatory. Our data was collected from the internet sources especially from those who have researched on the same topic before (refer to references).

\section{Conclusion}
Students should offer for Programs that best suite what they want to do especially in areas they are good at that is to say Arts or Sciences. Parents and teachers should see to it that they understand what is best for the students based on their performances in the subjects done at Lower Educational Levels.

\section{Recommendation}
Future researchers may perform a similar case study with multiple students and learning centres in order to compare results among. Future researchers may also study a small business which has considered its
social media efforts to be unsuccessful. Through an investigation of tracking and course recommendation strategy and the business’s, the research can identify why students may be struggling with grasping concepts in their respective degrees.

\title{Bibliography}
\author{}
\maketitle
\date{\today}

\bibliography{b}
\bibliographystyle{ieeetr}

\end{document}

\documentclass{article}

\begin{document}

\title{A MACHINE LEARNING APPROACH TO PERSONALIZING EDUCATION: IMPROVING INDIVIDUAL LEARNING THROUGH TRACKING AND COURSE RECOMMENDATION}
\maketitle

\subsection{Introduction}
This chapter provides a brief overview of the study presented in this thesis thereby introducing the reader to the scope, problem statement, significance
of the research, and research question, as well as the assumptions, limitations, and delimitations of the research.

\subsection{Background}
There are steadily increasing numbers of students enrolling in college and universities but the degrees offered ignore differences in backgrounds, abilities, learning styles and career goals. Worse more, due to insufficient teachers, students have a lot of work to  do by themselves. Poor learning results as well as low engagement, dissatisfaction and high dropout rates. In this project, an interactive electronic system will be built that is personalized for each student, is able to continuously track progress and goals, capitalize on the knowledge accumulated, and recommend suitable courses and activities in order to build skills, enhance interest and promote  long-term goals. In effect, our personalized interactive system operates as “if” there is a dedicated mentor for each student. To build this system, the following modules will need to be developed: (1) student and course similarity discovery methods; (2) student performance prediction algorithms; (3) personalized course recommendation algorithms.

\subsection{Objectives}
State briefly what are the expected outcomes of the research. b. State briefly how the results will be presented (info graphics, plots, tables etc).

\subsection{Scope}


\subsection{Problem Statement}
A problem statement should be presented within a context, and that context should be provided and briefly explained.

\subsection{Methodology}
We have used a literature review in our study. Our research is both descriptive and explanatory. Our data was collected from the internet sources especially from those who have researched on the same topic before (refer to references).


\subsection{Conclusion}
Companies should use social media in their marketing communications because it allows them to inform their customers and create a two-way communication. This communication can help companies to influence consumers and differentiate themselves. It can also help strengthen the corporate identity, build confidence for the company as well as create relationships. Social media is a cost effective way to become global and create reach.

\subsection{Recommendation}
Future researchers may perform a similar case study with multiple students and learning centres in order to compare results among . Future researchers may also study a small business which has considered its
social media efforts to be unsuccessful. Through an investigation of tracking and course recommendation strategy and the business’s, the research can identify why students may be struggling with grasping concepts in their respective degrees.

\subsection{References}

\subsubsection{Literature}
Alvesson M & Sköldberg K (2000), Reflexive Methodology – New Vistas for Qualitative Research, SAGE Publications, Great Britain.

\subsubsection{Scientific articles}
Goldner S (2010). "Take the A-path to Social Media Success". eContent Magazine, December 2010: 20-21.

\subsubsection{Internet sources}
Accessible: http://www.rohitbhargava.com/2011/01/the-top-15-marketing-social-media-trends-towatch- in-2011.html 

\end{document}